% !TeX spellcheck = en_US
\section{Introduction}
\label{sec_introduction}

To take advantage of multiple-core architecture of modern CPUs, an OS kernel is 
expected to handle concurrency. However, concurrent execution can inevitably 
introduce concurrency issues. Data race is one of the most harmful concurrency 
issues that occurs when multiple threads access a shared variable concurrently 
and at least one of the accesses is a write. Some races are harmful and can 
cause serious bugs like null-pointer dereferences, data inconsistencies and 
double fetches.

To detect data races effectively, some approaches~\cite{Boyapati:OOPSLA02, 
Anderson:PLDI08, Anderson:PLDI09, Zhou:MICRO19, Flanagan:PASTE01, 
Flanagan:PLDI00, Sadowski:PLATEAU14, ClangThreadSafety, Blackshear:OOPSLA18} 
rely on developers to supply annotations that describe the locking discipline 
such as which lock is required for accesses to a specific variable, and detect 
data races by searching variable accesses that violate these disciplines. For 
example, Clang thread safety analysis~\cite{ClangThreadSafety} requires 
developers to label a variable and the lock to protect it with a {\tt 
GUARDED\_BY} attribute. And then detects data races by finding variable 
accesses that violate the given attribute. However, such annotation-based 
approaches are not suitable for race detection in OS kernels, because even an 
expert developer can not provide accurate annotations, due to poor document and 
complicated logic of OS codes.

Other static approaches~\cite{Choi:PLDI02, Engler:SOSP03, Voung:FSE07, 
Pratikakis:PLDI06, Naik:PLDI06} employ lockset-based analysis to detect data 
races automatically. For a given variable, they first collect locks that are 
acquired for each access to it, and then detect data races by checking whether 
the intersection of locks for different accesses is empty. However, they do not 
consider alias relationships~\cite{Voung:FSE07, Engler:SOSP03} or just use 
imprecise flow-insensitive alias analysis~\cite{Choi:PLDI02, Pratikakis:PLDI06, 
Naik:PLDI06}, and thus can introduce both false positives and negatives.

Dynamic analysis can acquire run-time information such as memory address, and 
thus complex alias analysis can be avoided. Therefore, some 
approaches~\cite{Lochmann:EuroSys19, Lu:SOSP07, Lu:FSE18, Joshi:ASE08, 
Liu:NSDI07} detect data races dynamically to improve precision. They mine 
implicit code rules in softwares by statistically analyzing execution traces, 
and then use the mined rules to detect related bugs. For example, 
LockDoc~\cite{Lochmann:EuroSys19} records accesses to variables and lock 
acquisitions of an instrumented Linux kernel, and then infers locking rules 
from the recorded accesses. After that, LockDoc automatically locates variable 
accesses that violate the inferred locking rules to detect concurrency issues 
such as data races. However, dynamic analysis suffers from low code coverage, 
and thus the locking rules inferred through execution traces can be imprecise. 
Besides, dynamic analyses are hard to deploy, because they need to run OS 
kernels with instruments. At last, it is difficult for dynamic tools to 
estimate the harmfulness of a data race, because race conditions are hard to 
trigger by running the checked software~\cite{Fonseca:DSN10, 
Burckhardt:ASPLOS10, Liu:FSE14, Zhou:EASE15}.

In this paper, we design a practical static analysis approach named \sys, 
to automatically detect data races and estimate the harmfulness of these data 
races in OS kernels. \sys consists of three key techniques:

(T1) \sys uses an {\em alias-aware rule mining method} to automatically deduce 
locking rules about whether accesses to a specific data structure field should 
be protected and which data structure field the protecting lock exist in. We 
observe that the variable that is accessed and the lock to protect the access 
often exist in the same data structure. And this relationship between the 
accessed variable and the protecting lock can be effectively inferred from a 
special data structure named alias graph~\cite{Li:ASPLOS22, Kastrinis:CC18}. 
Specifically, variables in the same data structure have a common ancestor in 
the alias graph, and thus whether an accessed variable and a specific lock are 
in the same data structure can be inferred through finding a common ancestor in 
the alias graph. Moreover, with benefits from precise field-sensitive alias 
relationships of alias graph, \sys can effectively extract the accessed field 
and its corresponding lock field. After obtaining the accessed field and lock 
field pairs, \sys calculates the proportion of field accesses protected by a 
specific lock field in all field accesses. If the proportion is larger than a 
given threshold, the accessed field is determined to be protected by the lock 
field for a specific data structure.

(T2) \sys uses a {\em lock-usage analysis} to filter out false positives by 
validating concurrency of the kernel code. After mining locking rules, \sys 
detects data races by checking whether a given variable access violates the 
locking rules. To reduce false negatives, \sys conservatively assumes every two 
functions can execute concurrently, and thus can introduce false positives. 
However, we observe that each kernel module has an initialization phase, after 
which many functions can be called concurrently by other modules. When 
performing initialization, the kernel module serially initializes the locks and 
prepares other data for subsequent operations. And thus functions with the lock 
initialization and functions called by them tend not to execute concurrently. 
Based on this observation, \sys extracts all functions that are reachable from 
lock initialization functions such as {\tt spin\_lock\_init()} in the function 
call graph, and suppose that these functions can not execute concurrently.

(T3) \sys uses a {\em pattern-based estimation} to extract harmful races that 
can trigger memory or logic bugs such as null-pointer dereferences, data 
inconsistencies and double fetches. Some data races are benign, and developers 
do not put effort into repairing them because they can not cause serious memory 
or logic bugs. Therefore, it is important to estimate the harmfulness a data 
race can bring. We observe that harmful data races can be estimated by specific 
patterns. For example, if a raced data performs as an operand of a dereference 
instruction, a null-pointer dereference can occur; if a raced data structure is 
accessed for more than once and each access gets its different fields, a data 
inconsistency can occur; if a data race is caused by an unprotected write, and 
it is likely to introduce double fetches.Based on this observation, \sys 
exploits some patterns to extract harmful data races that can cause memory or 
logic bugs automatically.

We have implemented \sys with LLVM~\cite{clang} and Z3~\cite{z3}. 
\sys performs automated static analysis on the LLVM bytecode of the 
checked OS kernel. Overall, we make three main contributions in this paper:

\begin{itemize}
	\item We analyze the challenges to detect races in kernels, and propose 
	three key techniques to address	these challenges: (T1) an {\em alias-aware 
	rule mining method} to automatically deduce locking rules; (T2) a {\em 
	lock-usage analysis} to filter out false positives caused by code that can 
	not execute concurrently; (T3) a {\em pattern-based estimation} to extract 
	harmful data races that can cause memory or logic bugs such as null-pointer 
	dereferences, data inconsistencies and double fetches.		
	\item Based on these three key techniques, we design a practical static 	
	analysis approach named \sys, to effectively detect data races and 	
	estimate the harmfulness of these data races in OS kernels.
	\item We have evaluated \sys on Linux 6.2, and find 273 real data races, 	
	with a false positive rate of 19.9\%. Among these data races, 88 are 
	estimated to be harmful. We have reported these harmful bugs to Linux 
	kernel developers, and 32 of them have been confirmed.
\end{itemize}

The rest of this paper is organized as follows. Section~\ref{sec_motivation} 
introduces the motivation and challenges of data race detection in OS kernels. 
Section~\ref{sec_technique} introduces our key techniques to address these 
challenges. Section~\ref{sec_framework} introduces \sys. 
Section~\ref{sec_evaluation} shows our evaluation. Section~\ref{sec_discussion} 
makes a discussion about \sys. Section~\ref{sec_related} presents related 
work, and Section~\ref{sec_conclusion} concludes this paper.
