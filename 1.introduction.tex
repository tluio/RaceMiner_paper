% !TeX spellcheck = en_US
\section{Introduction}
\label{sec_introduction}

To take advantage of multiple-core architecture of modern CPUs, an OS kernel is 
expected to handle concurrency. However, concurrent execution can inevitably 
introduce concurrency issues. Data race is one of the most harmful concurrency 
issues that occurs when multiple threads access the same variable concurrently 
and at least one of the accesses is a write. Some data races are quite 
dangerous and can cause serious bugs such as null-pointer dereferences, data 
inconsistency and infinite loops.

To detect data races effectively, some approaches~\cite{Boyapati:OOPSLA02, 
Anderson:PLDI08, Anderson:PLDI09, Zhou:MICRO19, Flanagan:PASTE01, 
Flanagan:PLDI00, Sadowski:PLATEAU14, ClangThreadSafety, Blackshear:OOPSLA18} 
rely on developers to 
supply annotations that describe the locking discipline such as which lock 
protects a specific data structure field, and detect data races by exploiting 
data structure field accesses that violate these disciplines. For example, 
Clang thread safety analysis~\cite{ClangThreadSafety} requires developers to 
label a variable and the lock to protect it with a {\tt GUARDED\_BY} attribute. 
And then detects data races by finding variable accesses that violate the given 
attribute. However, such annotation-based approaches are not suitable for race 
detection in OS kernels, because even an expert developer can not provide 
accurate annotations, due to poor documentation and complicated logic of OS 
codes.

Some static approaches~\cite{Choi:PLDI02, Engler:SOSP03, Voung:FSE07, 
Pratikakis:PLDI06, Naik:PLDI06} employ lockset-based analysis to detect data 
races automatically. They first collect locks that are acquired when a given 
variable is accessed, and then mine locking rules about whether a given 
variable should be protected by a specific lock. At last, they detect data 
races by checking whether variable accesses violate the mined locking rules. 
However, they do not consider alias relations~\cite{Voung:FSE07, Engler:SOSP03} 
or just use imprecise flow-insensitive alias analysis~\cite{Choi:PLDI02, 
Pratikakis:PLDI06, Naik:PLDI06}, and thus can introduce both false positives 
and false negatives.

Dynamic analysis can acquire run-time information such as memory address, and 
thus complex alias analysis can be avoided. Therefore, some 
approaches~\cite{Lochmann:EuroSys19, Lu:SOSP07, Lu:FSE18, Joshi:ASE08, 
Liu:NSDI07} detect data races dynamically to improve precision. They mine 
implicit code rules in softwares by statistically analyzing execution traces, 
and then use the mined rules to detect related bugs. For example, 
LockDoc~\cite{Lochmann:EuroSys19} records accesses to data structure fields and 
lock acquisitions of an instrumented Linux kernel, and then infers locking 
rules from the recorded accesses. After that, LockDoc automatically locates 
data structure fields access that violate the inferred locking rules to detect 
concurrency bugs such as data races. However, dynamic analysis suffers from low 
code coverage, and thus the locking rules inferred through execution traces can 
be imprecise. Besides, dynamic analyses are hard to deploy, because they need 
to run OS kernels with instruments. At last, it is difficult for dynamic 
tools to estimate the harmfulness of a data race, because race conditions are 
hard to trigger by running the checked software~\cite{Fonseca:DSN10, 
Burckhardt:ASPLOS10, Liu:FSE14, Zhou:EASE15}.

In this paper, we design a practical static analysis approach named \sys, 
to automatically detect data races and estimate the harmfulness of these data 
races in OS kernels effectively. \sys consists of three key techniques:

(T1) \sys uses an {\em alias-aware rule mining method} to automatically 
deduce locking rules. We observe that an access to a data structure field is 
often protected by the lock in the same data structure as the accessed field. 
And this relationship between accessed field and the protecting lock can be 
effectively inferred from a special data structure named alias 
graph~\cite{Li:ASPLOS22, Kastrinis:CC18}. Specifically, fields in the same data 
structure have the same ancestor in the alias graph, and thus whether an 
accessed field and a specific lock are in the same data structure can be 
inferred by finding a common ancestor in the alias graph. Moreover, with 
benefits from precise field-sensitive alias relationships of alias graph, 
\sys can effectively extract the accessed field and its corresponding 
protecting lock. After obtaining the accessed field and protecting lock pairs, 
\sys calculates the proportion of field accesses protected by a specific 
lock in all field accesses. If the proportion is larger than a given threshold, 
the accessed field is determined to be protected by the protecting lock.

(T2) \sys uses a {\em lock-usage analysis} to filter out false positives 
validating concurrency of code paths. After mining locking rules, \sys detects 
data races by checking whether a given data structure field access violates the 
locking rules. To reduce false negatives, \sys conservatively assumes 
every two functions can be executed concurrently, and thus can introduce false 
positives. However, we observe that each kernel module has an initialization 
phase, after which many functions can be called concurrently by other modules. 
When performing initialization, the kernel module serially initialized the lock 
and prepare other data for subsequent operations. And thus functions with the 
lock initialization and functions called by them tend not to be executed 
concurrently. Based on this observation, \sys extracts all functions 
reachable from the function with a lock initialization operation such as {\tt 
spin\_lock\_init()}, and suppose that this function can not be executed in 
parallel.

(T3) \sys uses a {\em pattern-based estimation} to extract harmful races 
that can trigger memory or logical bugs such as null-pointer dereferences and 
data inconsistency. Some data races are benign, and developers do not put 
effort into repairing them because they can not cause serious memory or logical 
bugs. Therefore, it is important to estimate the harm a data race can bring. We 
observe that harmful data races can be estimated by given patterns. For 
example, if a raced data performs as an operand of a dereference instruction, a 
null-pointer dereference can occur; if a raced data is accessed for more than 
once and data inconsistency can occur. Based on this observation, \sys 
uses some patterns to extract harmful races that can cause memory or logical 
bugs automatically.

We have implemented \sys with LLVM~\cite{clang} and Z3~\cite{z3}. 
\sys performs automated static analysis on the LLVM bytecode of the 
checked OS kernel. Overall, we make three main contributions in this paper:

\begin{itemize}
	\item We analyze the challenges of data race detection in OS kernels, and 
	propose three key techniques to address	these challenges: (T1) an {\em 
	alias-aware rule mining method} to automatically deduce locking rules; (T2) 
	a {\em lock-usage analysis} to filter out false positives caused by 
	concurrency; (T3) a {\em pattern-based estimation} to extract harmful data 
	races that can trigger memory or logical bugs such as null-pointer 
	dereferences and data inconsistency.
	\item Based on these three key techniques, we design a practical static 
	analysis approach named \sys, to effectively detect data races and 
	estimate the harmfulness of these data races in OS kernels.
	\item We have evaluated \sys on Linux 6.2, and find xxx data races, 	
	with a false positive rate of xxx. Among these data races, xxx are 
	estimated to be harmful. We have reported these harmful bugs to Linux 
	kernel developers, and xxx of them have been confirmed.
\end{itemize}

The rest of this paper is organized as follows. Section~\ref{sec_motivation} 
introduces the motivation and challenges of data race detection in OS kernels. 
Section~\ref{sec_technique} introduces our key techniques to address these 
challenges. Section~\ref{sec_framework} introduces \sys. 
Section~\ref{sec_evaluation} shows our evaluation. Section~\ref{sec_discussion} 
makes a discussion about \sys. Section~\ref{sec_related} presents related 
work, and Section~\ref{sec_conclusion} concludes this paper.
