\documentclass[pageno]{jpaper}

%replace XXX with the submission number you are given from the ASPLOS submission site.
\newcommand{\asplossubmissionnumber}{XXX}

\usepackage[normalem]{ulem}
\usepackage{amsmath,amssymb,amsfonts}
\usepackage{xspace}
\usepackage{xstring}

\usepackage[numbers, sort]{natbib}

\newtheorem{definition}{Definition}
\newtheorem{example}{Example}

\newcommand{\sys}{\mbox{RaceMiner}\xspace}

\newcommand{\figcaption}{% 
	\setlength{\abovecaptionskip}{4pt}%	\setlength{\belowcaptionskip}{1pt}% 
	\caption}

\newcommand{\tablecaption}{% 
	\setlength{\abovecaptionskip}{6pt}%
	\setlength{\belowcaptionskip}{2pt}% 
	\caption}

\newcommand{\tabincell}[2]{\begin{tabular}{@{}#1@{}}#2\end{tabular}}

\newcommand{\PP}[1]{
	\vspace{2px}
	\noindent{\bf \IfEndWith{#1}{.}{#1}{#1.}}
}

\begin{document}

\title{
Static Race Detection in OS Kernels by Mining Locking Rules}

\date{}
\maketitle

\thispagestyle{empty}

\begin{abstract}
	
To take advantage of multiple-core architecture of modern CPUs, an operating 
system (OS) is expected to handle concurrency, inevitably introducing many 
concurrency issues. And data race is one of the most harmful concurrency issues 
that can trigger memory or logic bugs. Data races are caused by wrongly 
accessing shared data in different threads. To ensure correct access to shared 
data, fine-grained locking mechanisms are introduced. However, it is hard to 
determine whether a lock should be held when accessing a specific variable 
(locking rules) even for an expert developer, because of poor document and 
complicated logic of OS code. As a result, OS kernel is prone to data races, 
because necessary locks may be missed by mistake. Static analysis is a common 
technique to improve code quality, but it is quite challenging to detect data 
races in OS kernels automatically, due to lack of knowledge of locking rules 
and high complexity of concurrent execution.

In this paper, we design a practical static analysis approach named \sys, to 
effectively detect data races and estimate their harmfulness in OS kernels by 
mining locking rules. \sys first employs an alias-aware rule mining method to 
automatically deduce locking rules, and detects data races caused by violation 
of these rules. And then performs a lock-usage analysis to filter out false 
positives caused by code that can not execute concurrently. At last, \sys 
extracts harmful data races from all detected data races through a 
pattern-based estimation. We have evaluated \sys on Linux 6.2, and find 273 
data races, with a false positive rate of 19.9\%. Among these data races, 87 
are estimated to be harmful. We have reported these harmful data races to Linux 
kernel developers, and 32 of them have been confirmed.

\end{abstract}

% !TeX spellcheck = en_US
\section{Introduction}
\label{sec_introduction}

To take advantage of multiple-core architecture of modern CPUs, an OS kernel is 
expected to handle concurrency. However, concurrent execution can inevitably 
introduce concurrency issues. Data race is one of the most harmful concurrency 
issues that occurs when multiple threads access a shared variable concurrently 
and at least one of the accesses is a write. Some races are harmful and can 
cause serious bugs like null-pointer dereferences, data inconsistencies and 
double fetches.

To detect data races effectively, some approaches~\cite{Boyapati:OOPSLA02, 
Anderson:PLDI08, Anderson:PLDI09, Zhou:MICRO19, Flanagan:PASTE01, 
Flanagan:PLDI00, Sadowski:PLATEAU14, ClangThreadSafety, Blackshear:OOPSLA18} 
rely on developers to supply annotations that describe the locking discipline 
such as which lock is required for accesses to a specific variable, and detect 
data races by searching variable accesses that violate these disciplines. For 
example, Clang thread safety analysis~\cite{ClangThreadSafety} requires 
developers to label a variable and the lock to protect it with a {\tt 
GUARDED\_BY} attribute. And then detects data races by finding variable 
accesses that violate the given attribute. However, such annotation-based 
approaches are not suitable for race detection in OS kernels, because even an 
expert developer can not provide accurate annotations, due to poor document and 
complicated logic of OS codes.

Other static approaches~\cite{Choi:PLDI02, Engler:SOSP03, Voung:FSE07, 
Pratikakis:PLDI06, Naik:PLDI06} employ lockset-based analysis to detect data 
races automatically. For a given variable, they first collect locks that are 
acquired for each access to it, and then detect data races by checking whether 
the intersection of locks for different accesses is empty. However, they do not 
consider alias relationships~\cite{Voung:FSE07, Engler:SOSP03} or just use 
imprecise flow-insensitive alias analysis~\cite{Choi:PLDI02, Pratikakis:PLDI06, 
Naik:PLDI06}, and thus can introduce both false positives and negatives.

Dynamic analysis can acquire run-time information such as memory address, and 
thus complex alias analysis can be avoided. Therefore, some 
approaches~\cite{Lochmann:EuroSys19, Lu:SOSP07, Lu:FSE18, Joshi:ASE08, 
Liu:NSDI07} detect data races dynamically to improve precision. They mine 
implicit code rules in softwares by statistically analyzing execution traces, 
and then use the mined rules to detect related bugs. For example, 
LockDoc~\cite{Lochmann:EuroSys19} records accesses to variables and lock 
acquisitions of an instrumented Linux kernel, and then infers locking rules 
from the recorded accesses. After that, LockDoc automatically locates variable 
accesses that violate the inferred locking rules to detect concurrency issues 
such as data races. However, dynamic analysis suffers from low code coverage, 
and thus the locking rules inferred through execution traces can be imprecise. 
Besides, dynamic analyses are hard to deploy, because they need to run OS 
kernels with instruments. At last, it is difficult for dynamic tools to 
estimate the harmfulness of a data race, because race conditions are hard to 
trigger by running the checked software~\cite{Fonseca:DSN10, 
Burckhardt:ASPLOS10, Liu:FSE14, Zhou:EASE15}.

In this paper, we design a practical static analysis approach named \sys, 
to automatically detect data races and estimate the harmfulness of these data 
races in OS kernels. \sys consists of three key techniques:

(T1) \sys uses an {\em alias-aware rule mining method} to automatically deduce 
locking rules about whether accesses to a specific data structure field should 
be protected and which data structure field the protecting lock exist in. We 
observe that the variable that is accessed and the lock to protect the access 
often exist in the same data structure. And this relationship between the 
accessed variable and the protecting lock can be effectively inferred from a 
special data structure named alias graph~\cite{Li:ASPLOS22, Kastrinis:CC18}. 
Specifically, variables in the same data structure have a common ancestor in 
the alias graph, and thus whether an accessed variable and a specific lock are 
in the same data structure can be inferred through finding a common ancestor in 
the alias graph. Moreover, with benefits from precise field-sensitive alias 
relationships of alias graph, \sys can effectively extract the accessed field 
and its corresponding lock field. After obtaining the accessed field and lock 
field pairs, \sys calculates the proportion of field accesses protected by a 
specific lock field in all field accesses. If the proportion is larger than a 
given threshold, the accessed field is determined to be protected by the lock 
field for a specific data structure.

(T2) \sys uses a {\em lock-usage analysis} to filter out false positives by 
validating concurrency of the kernel code. After mining locking rules, \sys 
detects data races by checking whether a given variable access violates the 
locking rules. To reduce false negatives, \sys conservatively assumes every two 
functions can execute concurrently, and thus can introduce false positives. 
However, we observe that each kernel module has an initialization phase, after 
which many functions can be called concurrently by other modules. When 
performing initialization, the kernel module serially initializes the locks and 
prepares other data for subsequent operations. And thus functions with the lock 
initialization and functions called by them tend not to execute concurrently. 
Based on this observation, \sys extracts all functions that are reachable from 
lock initialization functions such as {\tt spin\_lock\_init()} in the function 
call graph, and suppose that these functions can not execute concurrently.

(T3) \sys uses a {\em pattern-based estimation} to extract harmful races that 
can trigger memory or logic bugs such as null-pointer dereferences, data 
inconsistencies and double fetches. Some data races are benign, and developers 
do not put effort into repairing them because they can not cause serious memory 
or logic bugs. Therefore, it is important to estimate the harmfulness a data 
race can bring. We observe that harmful data races can be estimated by specific 
patterns. For example, if a raced data performs as an operand of a dereference 
instruction, a null-pointer dereference can occur; if a raced data structure is 
accessed for more than once and each access gets its different fields, a data 
inconsistency can occur; if a data race is caused by an unprotected write, and 
it is likely to introduce double fetches.Based on this observation, \sys 
exploits some patterns to extract harmful data races that can cause memory or 
logic bugs automatically.

We have implemented \sys with LLVM~\cite{clang} and Z3~\cite{z3}. 
\sys performs automated static analysis on the LLVM bytecode of the 
checked OS kernel. Overall, we make three main contributions in this paper:

\begin{itemize}
	\item We analyze the challenges to detect races in kernels, and propose 
	three key techniques to address	these challenges: (T1) an {\em alias-aware 
	rule mining method} to automatically deduce locking rules; (T2) a {\em 
	lock-usage analysis} to filter out false positives caused by code that can 
	not execute concurrently; (T3) a {\em pattern-based estimation} to extract 
	harmful data races that can cause memory or logic bugs such as null-pointer 
	dereferences, data inconsistencies and double fetches.		
	\item Based on these three key techniques, we design a practical static 	
	analysis approach named \sys, to effectively detect data races and 	
	estimate the harmfulness of these data races in OS kernels.
	\item We have evaluated \sys on Linux 6.2, and find 273 real data races, 	
	with a false positive rate of 19.9\%. Among these data races, 88 are 
	estimated to be harmful. We have reported these harmful bugs to Linux 
	kernel developers, and 32 of them have been confirmed.
\end{itemize}

The rest of this paper is organized as follows. Section~\ref{sec_motivation} 
introduces the motivation and challenges of data race detection in OS kernels. 
Section~\ref{sec_technique} introduces our key techniques to address these 
challenges. Section~\ref{sec_framework} introduces \sys. 
Section~\ref{sec_evaluation} shows our evaluation. Section~\ref{sec_discussion} 
makes a discussion about \sys. Section~\ref{sec_related} presents related 
work, and Section~\ref{sec_conclusion} concludes this paper.

% !TeX spellcheck = en_US
\section{Motivation}
\label{sec_motivation}

\subsection{A Motivating Example}
\label{subsec_motivating_example}

\begin{figure}[htbp]
	\centering
	\includegraphics[width=1\linewidth]{figures/fig_demo_bug.pdf}
	\figcaption{A null-pointer dereference due to data race in Linux 6.2.}
	\label{fig_demo_bug}
\end{figure}

Figure~\ref{fig_demo_bug} shows a real null-pointer dereference caused by a 
data race in the Linux DRM driver. In the DRM driver, the functions {\tt 
exynos\_drm\_crtc\_atomic\_disable()} and {\tt exynos\_crtc\_handle\_event()} 
can execute concurrently. We exploit \textcircled{\footnotesize{n}} to 
represent the execution order of instructions and show one execution case on 
the left of Figure~\ref{fig_demo_bug}. In Thread 1, the variables {\tt 
crtc->state->event} and {\tt crtc->state->active} are checked by an if 
statement in the function {\tt exynos\_drm\_crtc\_atomic\_disable()}. After the 
condition is calculated to be true, the function {\tt 
exynos\_crtc\_handle\_event()} is executed in Thread 2. In this function, the 
value of {\tt crtc->state->active} is assigned to {\tt event} 
(\textcircled{\footnotesize{2}}) and then {\tt event} is checked in an if 
statement (\textcircled{\footnotesize{3}}). If it is not NULL, the variable 
{\tt crtc->state->event} is assigned with NULL 
(\textcircled{\footnotesize{4}}). Right after this assignment, the function 
{\tt drm\_crtc\_send\_vblank\_event} is called  in Thread 1 
(\textcircled{\footnotesize{6}}) with the argument {\tt crtc->state->active}, 
after acquiring the lock {\tt crtc->dev->event\_lock} 
(\textcircled{\footnotesize{5}}). In the called function, the variable {\tt 
crtc->state->event} is dereferenced through {\tt e->pipe} 
(\textcircled{\footnotesize{6}}). In this execution case, the data structure 
{\tt crtc->state->event} is first assigned with NULL and then dereferenced, and 
thus a null-pointer dereference can occur.

This bug is triggered only when {\tt crtc->state->event} is set to NULL by 
Thread 2 right after the first condition of the if statement in Thread 1 is 
calculated to be true. Such a requirement is difficult to satisfy by executing 
existing test suites. In fact, this bug had existed for nearly 6 years since 
Linux 4.14 (Released in Nov. 2017), and it was fixed by us based on a report 
generated by \sys. 
 
\subsection{Challenges}
\label{subsec_challenges}
Detecting data races and estimating their harmfulness in OS kernels have three 
main challenges:

\PP{C1: Getting locking rules.} The relationship between variables and locks 
is not well documented in OS kernels, making it hard to determine whether a 
specific variable should be protected by a lock and which lock is 
required (locking rules), even for an expert developer. And thus existing 
annotation-based approaches~\cite{Boyapati:OOPSLA02, Anderson:PLDI08, 
Anderson:PLDI09, Zhou:MICRO19, Flanagan:PASTE01, Flanagan:PLDI00, 
Sadowski:PLATEAU14, ClangThreadSafety, Blackshear:OOPSLA18} are difficult to 
apply to race detection in OS kernels. Other approaches~\cite{Choi:PLDI02, 
Engler:SOSP03, Voung:FSE07, Pratikakis:PLDI06, Naik:PLDI06} employ 
lockset-based analysis to detect data races automatically, but they do not 
consider alias relations~\cite{Voung:FSE07, Engler:SOSP03} or just use 
imprecise flow-insensitive alias analysis~\cite{Choi:PLDI02, 	
Pratikakis:PLDI06, Naik:PLDI06}. However, due to the heavy use of pointers and 
data structure fields in OS code, the alias relationships between variables can 
be very complex, and thus lacking effective alias analysis can introduce many 
false locking rules.

\PP{C2: Dropping false data races.} Static analysis suffers from false 
positives. For example, each data race involves more than one code path that 
should be able to concurrently execute. However, which code can execute 
concurrently is not well documented for an OS kernel, and it is also hard to 
determine concurrent code statically due to the complexity of OS code. Thus 
static analysis can report many false data races.

\PP{C3: Estimating harmfulness of data races.} Many data races are benign, and 
can not cause memory or logic bugs, and thus developers are unwilling to put 
effort into repairing them. To automatically detect harmful data races, most 
approaches~\cite{Narayanasamy:PLDI07, Sen:PLDI08, Kasikci:SOSP13, 
Kasikci:ASPLOS12} estimate the harmfulness of data races through dynamic 
analysis, and explore thread interleavings to trigger data races to estimate 
their harmfulness. However, they suffer from low code coverage and thus can 
miss many real harmful data races.



% !TeX spellcheck = en_US
\section{Race Detection by Mining Locking Rules}
\label{sec_technique}
To address the above challenges, we propose three key techniques. For {\em C1}, 
we propose an {\em alias-aware rule mining method} to automatically deduce 
locking rules. For {\em C2}, we propose a {\em lock-usage analysis} to filter 
out false data races by validating concurrency of code paths. For {\em C3}, we 
propose a pattern-based estimation to extract harmful races that can trigger 
memory or logical bugs such as null-pointer dereference and data inconsistency. 
We introduce them as follows:

\subsection{Alias-Aware Rule Mining Method}
\label{subsec_rule_mining}
The relationship between variables and locks are not well documented in OS 
kernels, but it can be inferred from the kernel code. Specifically, a data 
structure field is often protected by the lock stored in the same data 
structure. And thus if a data structure field is accessed after acquiring a 
lock existing in the same data structure in most cases, the variable is likely 
to be protected by the lock. Whether a data structure field and the protecting 
lock exist in the same data structure can be determined though an alias 
graph~\cite{Li:ASPLOS22, Kastrinis:CC18} by finding their common ancestor. 
Based on this insight, we propose an {\em alias-aware rule mining method} to 
deduce locking rules automatically. Moreover, with benefits from precise 
field-sensitive alias relationships of alias graph, our alias-aware rule mining 
method can find data structure filed and its protecting lock effectively.

\PP{Alias Graph.} It is an important data structure to infer relationships 
between data structure field and its protecting lock in our analysis, so we 
introduce it and its update first. 

An alias graph is a 2-tuple $\mathit{G = \left<N, E\right>}$, where 
$\mathit{N}$ is a set of nodes, and each node $\mathit{n}$ represents an alias 
set that points to one abstract object. $\mathit{E}$ is a set of labeled edges. 
Each edge is labeled with a data structure field or a dereference operator 
``$\mathit{*}$'', which represents how an abstract object is accessed.

An alias graph is updated by handling four types of instructions that  
change alias relationships: MOVE($\mathit{v_1 = v_2}$), STORE ($\mathit{*v_2 = 
v_1}$), LOAD ($\mathit{v_1 = *v_2}$) and GEP ({$\mathit{v_1 = \&v_1->f}$}). We 
exploit {$\mathit{n_x}$} to represent the node whose representing alias set 
includes $\mathit{v_x}$, and introduce how the four types of instructions 
update alias graphs. For a MOVE operation, $\mathit{v_1}$ is moved from 
$\mathit{n_1}$ to $\mathit{n_2}$. After this operation, $\mathit{v_1}$ and  
$\mathit{v_2}$ are represented by the same node, which indicates they become 
aliases. For a STORE operation, the existing outgoing edge from $\mathit{n_2}$ 
is dropped first, and then a new edge labeled with $\mathit{*}$ from 
$\mathit{n_2}$ to $\mathit{n_1}$ is inserted. After this operation, 
$\mathit{v_1}$ and $\mathit{*v_2}$ are represented by the same node, which 
indicates they become aliases. For a LOAD operation, the analysis first finds 
the destination node of the edge that comes from $\mathit{n_2}$ and is labeled 
with $\mathit{*}$, and then move $\mathit{v_1}$ to the destination node. And 
after this operation, $\mathit{v_1}$ and $\mathit{*v_2}$ are represented by the 
same node, which indicates they become aliases. GEP operation is similar to 
LOAD, expect that the edge is labeled with a data structure field $\mathit{f}$, 
instead of a dereference operator $\mathit{*}$.

\begin{figure}[htbp]
	\centering
	\includegraphics[width=0.9\linewidth]{figures/fig_alias_graph_demo.pdf}
	\figcaption{Example of alias graph.}
	\label{fig_alias_graph_demo}
\end{figure}

\noindent{\textbf{\em Example.}} Figure~\ref{fig_alias_graph_demo} shows a 
piece of driver-like source code and its alias graph. In this example, after an 
GEP (\&dev->lock) operation at Line 6, an edge labeled with {\tt lock} from 
node $\mathit{n_1}$ to node $\mathit{n_4}$ is inserted. Similarly, an edge 
labeled with {\tt flag} from node $\mathit{n_1}$ to node $\mathit{n_2}$ is 
inserted after Line 7. At last, an edge labeled with a dereference operator 
($\mathit{*}$) is inserted after the STORE (*fp = f) operation at Line 8. In 
this example, {\tt \&dev->flag} is represented by node $\mathit{n_2}$, and {\tt 
\&dev->lock} is represented by node $\mathit{n_4}$. The two nodes have a common 
ancestor $\mathit{n_1}$, and thus the accessed data structure field {\tt 
\&dev->flag} and the protecting lock {\tt \&dev->lock} can be inferred to exist 
in the same data structure (namely {\tt Dev}).
% !TeX spellcheck = en_US
\section{Framework}
\label{sec_framework}

% !TeX spellcheck = en_US
\section{Evaluation}
\label{sec_evaluation}

To validate the effectiveness of \sys, we evaluate it on the code of Linux 
kernel 6.2. We run the evaluation on a regular x86-64 desktop with sixteen 
Intel i7-10700 CPU@2.90GHz processors and 64GB physical memory. We use the 
kernel configuration {\em allyesconfig} to enable all kernel code for the 
x86-64 architecture.

\begin{table}[tbph]
	\tablecaption{Detection results of Linux 6.2.}
	\label{tbl_bug_detection}
	\renewcommand{\arraystretch}{1}
	\setlength\tabcolsep{2pt}
	\noindent{\scriptsize
		\begin{center}
			\begin{tabular}{p{1.5cm}|l|c}
				\hline
				\multicolumn{2}{c|}{\textbf{Description}} & \textbf{\sys}  
				\\ \hline
				\multirow{2}{1.5cm}{\textbf{{\em Code analysis}}} & 
				Source files (analyzed/all) & xxxK/xxxK
				\\ \cline{2-3}
				& Source code lines (analyzed/all) & xxxM/xxxM 
				\\ \cline{2-3}
				\hline
				\multirow{3}{1.5cm}{\textbf{{\em Locking-rule mining}}} & 
				Key fields / total variables& 
				\\ \cline{2-3}
				& Mined locking rules &
				\\ \cline{2-3}
				& Protected field accesses / all accesses & 0.6
				\\ \cline{2-3}
				\hline
				\multirow{2}{1.5cm}{\textbf{{\em Data race detection}}} & 
				Detected data races (real / all) & 273 / 341
				\\ \cline{2-3}
				& Dropped data races by lock-usage analysis & 63
				\\ \cline{2-3}
				\hline
				\multirow{5}{1.5cm}{\textbf{{\em Data race estimation}}}
				& Null-pointer dereference (confirmed / all) & 15 / 20
				\\ \cline{2-3}
				& Infinite loop (confirmed / all) & 0 / 1
				\\ \cline{2-3}
				& Data inconsistency (confirmed / all) & 7 / 10
				\\ \cline{2-3}
				& Unprotected write (confirmed / all) & 10 / 57
				\\ \cline{2-3}
				& Total harmful data races (confirmed / all) & 32 / 88
				\\ \cline{2-3}
				\hline
				\multirow{3}{1.5cm}{\textbf{{\em Time usage}}} & 
				Key-field extraction & 
				\\ \cline{2-3}
				& Data-race detection &
				\\ \cline{2-3}
				& Total time &
				\\ \cline{2-3}
				\hline
			\end{tabular}
	\end{center}}
\end{table}

\subsection{Bug Detection}
\label{subsec_bug_detection}

We configure \sys with common lock-acquiring/release functions (like {\tt 
spin\_lock and spin\_unlock}) to perform lock-set analysis to extract key 
fields and mine locking rules, and lock-initialization functions (like {\tt 
spin\_lock\_init}) to filter our false data races caused by code paths that 
can not execute concurrently. And then run \sys to automatically check the 
kernel source code. We manually check all the data races found by \sys, and 
Table~\ref{tbl_bug_detection} shows the results, and source code lines are 
counted by CLOC~\cite{cloc}. From the results, we have the following findings:

\PP{Code analysis.} \sys can scale to large code bases of OS kernels, and it in 
total analyze xxxM lines of code in xxxK source code files within xxx hours. 
The remaining xxxM lines of code in xxxK source files are not analyzed, as they 
are not enabled by the {\em allyesconfig} for the x86-64 architecture. We 
believe that \sys can also find more data races in other architectures with 
proper configuration.

\PP{Locking-rule mining.} An OS kernel has a large code base with numerous 
variables. Handling all variables when mining locking rules can introduce much 
overhead. However, we observe that a variable tend to be protected by the lock 
stored in the same data structure as the accessed variable. Based on this 
observation, our locking-rule mining method first extract a key field by 
finding whether there exists any access to it that is protected by a lock 
stored in the same data structure. This method drops xxx\% variables (xxx out 
xxx) that need to handled when mining locking rules, and thus can reduce 
overhead significantly. After extracting key fields, our locking-rule mining 
method collect all accessed to these key fields, and then deducing locking-rule 
based on statistic. In this paper, given a data structure field, we set the 
threshold of the ratio of accesses protected by a specific lock to all accesses 
to 0.6. Our alias-aware rule mining method can drop many false rules and thus 
only mines xxx rules, which can effectively reduce false data races.

\PP{Data race detection.} \sys reports 341 data races in the kernel source 
code. We spent 15 hours on checking these data races and identify that 273 of 
them are real, with a false positive rate of 19.9\%. Besides, our lock-usage 
analysis drops 63 false data races and the results shows that it can reduce 
false positives significantly.

\PP{Data race estimation.} Many data races are benign and can not cause memory 
or logic bugs, and thus developers are unwilling to put effort into repairing 
them. We exploit four patterns to detect null-pointer dereferences, infinite 
loop, data inconsistency and unprotected write as introduced in 
Section~\ref{subsec_estimation}, and find 88 data races in these patterns. We 
report them to developers and 32 of them have been confirmed and fixed by them. 
We still wait for response of other data races. Moreover, one of the developers 
wonders if \sys can be used in their CI to detect these problems. 
The results show that our pattern-based estimation can extract harmful data 
races effectively, and can considerably reduce the workload of developers.

\subsection{False Positives and Negatives}
\label{subsec_false_pos_neg}
\sys reports 68 false data races, and through manually checking these false 
data races, we find that they are introduced for three main reasons:

First, \sys employs an alias-aware rule mining method to deduce locking rules. 
To improve the precision of mined rules, it assumes that the accessed variable 
and the protecting lock are stored in the same data structure. Even so, \sys 
can also deduce some false rules because some developers does not use locks 
properly. They take a lock/unlock pair to protect accesses to all fields in the 
same data structure, instead of the exact field should be protected for 
convenience. As a result, our alias-aware rule mining method infers that 
accesses to all fields surrounded by the lock/unlock pair should be protected 
by the lock. This reason causes \sys to report 47 false data races.

\begin{figure}[htbp]
	\centering
	\includegraphics[width=1\linewidth]{figures/fig_demo_false_rule.pdf}
	\figcaption{A false data race caused by an incorrect locking rule.}
	\label{fig_demo_false_rule}
\end{figure}

Figure~\ref{fig_demo_false_rule} shows a false data race caused by an incorrect 
locking rule. In this example, only accesses to {\em cmd->t\_state} and {\em 
cmd->transport\_state} should be protected by the lock {\em 
cmd->t\_state\_lock}. However, the lock operation is put ahead of the switch 
statement by developers, making our alias-aware rule mining method deduces that 
{\em cmd->scsi\_status} and {\em cmd->se\_cmd\_flags} also need to be protected 
by {\em cmd->t\_state\_lock} mistakenly. Based on this incorrect locking rule, 
\sys reports a false positive at Line 745 when {\em cmd->se\_cmd\_flags} is 
accessed in an if statement. Although this data race is a false positive, it 
can cause performance degradation because the critical zone should have been 
limited to Lines 899-900.

Second, in order to reduce memory overhead and improve the performance of data 
passing among different functions, an integer can be divided into several bit 
vector to represent different data structure fields. However, 

% !TeX spellcheck = en_US
\section{Discussion}
\label{sec_discussion}

% !TeX spellcheck = en_US
\section{Related Work}
\label{sec_related}

% !TeX spellcheck = en_US
\section{Conclusion}
\label{sec_conclusion}

In this paper, we develop a novel static approach named \sys to detect data 
races in OS kernels by mining locking rules. It consists of three key 
techniques, including an alias-aware rule mining method to automatically deduce 
locking rules, a lock-usage analysis to filter out false positives caused by 
code that can not execute concurrently and a pattern-based estimation to 
extract harmful data races that can trigger memory or logic bugs such as 
null-pointer dereferences, data inconsistencies and double fetches. In the 
evaluation, \sys finds 87 real harmful data races, and 32 of them have been 
confirmed by the developers.

\footnotesize
\bibliographystyle{plain}
\bibliography{references}


\end{document}

